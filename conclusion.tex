\begin{comment}

\addcontentsline{toc}{section}{Conclusion}
\chapter{Conclusion}
 the relationship between travel length, speed and droplet formation 
The mechanical setup using Thorlabs works satisfactory. The merge between Thorlabs hardware and other hardware has also been implemented with success. There are however some hardware-related issues that affects the measurements, and limits the possibilities for complete testing and software development. 

A software for measurements of volume in a droplet has been created, and it produces results that to some extent gives plausible results. We do however not have the knowledge or equipment to cross check these results and perform volume measurements of a droplet to compare with our results. As mentioned in the chapter Testing and Results the droplet width was sometimes measured to be significantly larger than the actual size. Since the width of the fiber cable is fixed with a width of 125 $\mu$ m the droplet width can not have a width higher than this. If we set the width of the fiber cable fixed at 125 $\mu$ m, the process of masking this area is simplified, but it doesn't explain the large variances seen in the droplet volume as it is only one factor in the measurements. 

In order to produce results that are scientifically more accurate one would have to do changes to both hardware and software. The camera we used for our measurements does not have auto focus. This means that for each sample the camera has to be manually focused. This has been done by moving the camera on the x- and y-axis. This process changes the distance between the camera and the fiber cable, and the uncertainty of volume measurements are greatly increased as a consequence. Obtaining a camera with auto focus would greatly increase the chances of being able to reproduce results. 

Provided the changes to the mechanical setup is implemented, an automatized test routine could be implemented. The test platform described in chapter 3 could be used for this purpose. In this scenario a mask would not be needed in between each sample measurement, this would give a higher accuracy of measurements. This would also enable extensive series testing. This would give a better indicator of the accuracy of the system, and a better basis for research on the relationship between travel length, speed and droplet formation.  

The masking of the droplet is essential in producing an accurate measurement of volume. The current method should be enhanced. 

\end{comment}