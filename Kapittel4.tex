\begin{comment}
\chapter{Testing and results}

All of the droplet volume tests has been performed according to the step by step manual for the user, wich can be found in appendix \ref{cha:Test procedure}. This chapter will present additional elements in the test procedure, present the test results and discuss them. 

Originally the fiber should be immersed in a beacer of two liquid solutions: Squalane oil and distilled water. Squalane oil is lighter than distilled water, and we would therefore measure the size of the water droplet when it came from the distilled water into squalane oil.
Measuring the size of a droplet through the beacer proved to be a difficult task, partly because of the refractive index in the beacer and the distribution of light. 
It was therefore tried to measure the droplet size of distilled water into the air, but since the drop has time to evaporate before a picture is taken, this is not feasible at this time. It was therefore decided that we measure the droplet size of squalane oil in air,since this oil doesn't evaporate at the same rate as distilled water. It is not the best solution in relation to use in real life, but is currently the best solution to carry out final measurements.


The tests have been performed with four acceleration values: 0.5 , 1.0 , 1.5 and 2.0   $mm/s^2$

The tests have been conducted in three separate series, where the goal was to see if there is any correlation of speed and droplet volume.

The same start and stop position has been used for all attempts in each serie, but the three series have different start- and stop positions, resulting in varying travel length for the fiber. 
Travel length serie 1: 1,3 mm
Travel length serie 2: 2,0 mm
Travel length serie 3: 2,0 mm


\section{Test results}
Complete test results from test series 1-3 can be found in Appendix \ref{cha:test_res}. On the next page one can see graphical displays of the results from the test series. 



\begin{figure}[h!]
	\centering
	\includegraphics[scale=0.65]{pictures/FinVolTest/Combined/Acc0_5L}
	\caption{Three test series. Acceleration: $0.5$ $mm/s^2$}
	\label{img:Acc0_5}
\end{figure}

\begin{figure}[h!]
	\centering
	\includegraphics[scale=0.65]{pictures/FinVolTest/Combined/Acc1_0L}
	\caption{Three test series. Acceleration: $1.0$ $mm/s^2$}
	\label{img:Acc1_0}
\end{figure}

\begin{figure}[h!]
	\centering
	\includegraphics[scale=0.65]{pictures/FinVolTest/Combined/Acc1_5L}
	\caption{Three test series. Acceleration: $1.5$ $mm/s^2$}
	\label{img:Acc1_5}
\end{figure}

\begin{figure}[h!]
	\centering
	\includegraphics[scale=0.65]{pictures/FinVolTest/Combined/Acc2_0L}
	\caption{Three test series. Acceleration: $2.0$ $mm/s^2$}
	\label{img:Acc2_0}
\end{figure}


\newpage

\section{Comments}
Figure \ref{img:Acc0_5} to \ref{img:Acc2_0} shows that there is no correlation in the test results between the three series.



During the test procedure we experienced that the measured volume changed more than expected, and that the two measurements that would give the same results were not near each other. The width of a droplet could be measured to be approximately 4000 um, a size that is clearly not possible to achieve. Those results are considered corrupted and are thus neglected.

The volume of the droplet was also calculated with a fixed width of 125 um for some of the datasets, but with no good result. The data points in the curve is more evenly distributed, but the curve still doesnt have the desired curvature.

\begin{figure}[!ht]
	\centering
	\includegraphics[width=0.7\textwidth]{pictures/FinVolTest/Vol_m_vs_f_width}
	\caption{Comparison: Calculated droplet volume with measured width and fixed width}
	\label{img:m_vs_f}
\end{figure}

\end{comment}