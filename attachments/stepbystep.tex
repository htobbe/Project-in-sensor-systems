\begin{comment}
\chapter{Step by Step manual for the user} \label{cha:Test procedure}

\section{Step by Step manual for the user} 

\begin{figure}[h!]
	\centering
	\includegraphics[scale=0.4]{UserInterface}
	\caption{User Interface}
	\label{img:UserInterface}
\end{figure} 

\newpage
\renewcommand{\labelenumi}{\Roman{enumi}.}
\begin{enumerate}
\item Select the camera name in the User interface which should be used for the droplet detection. 
\item Select the COM-port which is used for the light control.
\item Run the VI.
\item Drive the fiber-cable in a fix exact position in the scope of the camera by using Thorlabs APT User. Remember the position.
\item Adjust the focus for sharp edges. 
\item Choose the threshold options:

\begin{itemize}
\item Look for: Dark options
\item Method: Background
\item Window Size: 128x128
\end{itemize}
							
\item Press "Detect without droplet". (This will save the picture on your disk and the mask will be created)
\item Drive the fiber-cable down to create a droplet and drive it back exactly in the remembered position. 
\item Press "Detect with droplet" (This picture will be saved)
\item Press "Use Mask"
\item Press "Find particles" (Calculation of the volume) 
\end{enumerate}
\end{comment}