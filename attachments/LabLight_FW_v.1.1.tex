\chapter{Controller board firmware} \label{cha:ERM_APX1}


\lstset{language=C++,
        basicstyle=\scriptsize\ttfamily,
        keywordstyle=\scriptsize\color{blue}\ttfamily,
        stringstyle=\scriptsize\color{red}\ttfamily,
        commentstyle=\scriptsize\color{black}\ttfamily,
        morecomment=\scriptsize[l][\color{magenta}]{\#}
}

\section{Main function} \label{cha:ERM_APX1_1}

\begin{lstlisting}

#include <stdio.h>
#include <avr/io.h>
#include <avr/interrupt.h>

#define F_CPU 11059200UL    // External 11,0592 MHz oscillator fuse activated
#include <util/delay.h>

int main(void){
    uart_init(F_CPU,57600);     // Initialize UART communication, 57600 baud
    pwm_init();                 // Initialize PWM
    set_bright(0);              // Turn off LEDs
    sei();                      // Enable global interrupts
    
    while(1){
        // Do nothing
    }
}

ISR(USART1_RX_vect){    // Interrupt routine triggered by received USART communication
    char temp;
    temp = UDR1;        // Store received character
    set_bright(temp);   // Set new duty cycle
    UDR1 = temp;        // Echo back received character
}

\end{lstlisting}
\clearpage


\section{UART initialization and send function} \label{cha:ERM_APX1_2}

\begin{lstlisting}[language=C++]

#include <avr/io.h>	

void uart_init(uint32_t f_cpu,uint32_t baud);
void uart_send(char *str);

void uart_init(uint32_t f_cpu, uint32_t baud){
    uint16_t baud_rate = ((f_cpu/(16*baud))-1);
    UCSR1B |= ((1<<RXEN1)|(1<<TXEN1));          // Enable UART receiver and transmitter
    UBRR1H |= baud_rate >> 8;                   // Set baud rate high byte
    UBRR1L |= baud_rate;                        // Set baud rate low byte
    UCSR1B |= (1<<RXCIE1);                      // Enable receiver interrupt
}

void uart_send(char *str){
    while (*str){
        while (!(UCSR1A & (1<<UDRE1))){
            // Wait for transmission to end
        }
        UDR1 = *str;        // Transmit character
        str++;              // Advance string position
    }
}

\end{lstlisting}
\clearpage


\section{PWM initialization and set brightness function} \label{cha:ERM_APX1_3}

\begin{lstlisting}[language=C++]

#include <avr/io.h>

void pwm_init(void);
void set_bright(uint8_t bright);

void pwm_init(void){
    DDRB |= (1<<PORTB3);                    // Set LED1 (PB3) as output
    TCCR0A |= ((1<<WGM01)|(1<<WGM00));      // Initialize fast PWM mode
    TCCR0A |= (1<<COM0A1);                  // Clear PB3 on compare match, 
                                            // set on BOTTOM (non-inverting mode)
    TCCR0B |= ((1<<CS01)|(1<<CS00));        // Set 1/64 prescaler
}

void set_bright(uint8_t bright){
    if (bright == 0){
        TCCR0A = 0;                         // Disable PWM
    }
    else{
        pwm_init();                         // Reenable PWM
        OCR0A = bright;                     // Set counter to clear on bright
    }
}

\end{lstlisting}
\clearpage